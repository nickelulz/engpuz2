\documentclass[11pt]{article}
\usepackage{enumitem}
\usepackage{tgschola}

\usepackage{titling}
\renewcommand\maketitlehooka{\null\mbox{}\vfill}
\renewcommand\maketitlehookd{\vfill\null}

\setlength\parindent{0pt}

\title{The Secret History Puzzle 2}
\author{Mufaro Machaya and Mark Haddadin}

\date{March 15, 2024}

\begin{document}

\begin{titlingpage}
\maketitle
\end{titlingpage}

\textit{You've already made it this far, so I must confess, I've been keeping great secrets from you. I broke up the truth into two parts and hid it in just the way you might expect, with the nuances of language. Take the number value of each excerpt from my diary and match it to the literary term.} \\

\section{Excerpts}

\begin{minipage}{\textwidth}
\begin{enumerate}[series=main]
  \item “Why does that obstinate little voice in our heads torment us so? Could it be because it reminds us that we are alive, of our mortality, of our individual souls – which, after all, we are too afraid to surrender but yet make us feel more miserable than any other thing? It is a terrible thing to learn as a child that one is a being separate from the world, that no one and no thing hurts along with one’s burned tongues and skinned knees, that one’s aches and pains are all one’s own. Even more terrible, as we grow older, to learn that no person, no matter how beloved, can ever truly understand us. Our own selves make us most unhappy, and that’s why we’re so anxious to lose them, don’t you think?” \\
\end{enumerate}
\end{minipage}

\begin{minipage}{\textwidth}
\begin{enumerate}[resume=main]
  \item “Could it be because it reminds us that we are alive, of our mortality, of our individual souls- which, after all, we are too afraid to surrender but yet make us feel more miserable than any other thing? But isn't it also pain that often makes us most aware of self? It is a terrible thing to learn as a child that one is a being separate from the world, that no one and no thing hurts along with one's burned tongues and skinned knees, that one's aches and pains are all one’s own. Even more terrible, as we grow old, to learn that no person, no matter how beloved, can ever truly understand us. Our own selves make us most unhappy, and that's why we're so anxious to lose them, don't you think? \\
\end{enumerate}
\end{minipage}

\begin{minipage}{\textwidth}
\begin{enumerate}[resume=main]
  \item “It's a very Greek idea, and a very profound one. Beauty is terror. Whatever we call beautiful, we quiver before it. And what could be more terrifying and beautiful, to souls like the Greeks or our own, than to lose control completely? To throw off the chains of being for an instant, to shatter the accident of our mortal selves? Euripides speaks of the Maenads: head thrown I back, throat to the stars, "more like deer than human being." To be absolutely free! One is quite capable, of course, of working out these destructive passions in more vulgar and less efficient ways. But how glorious to release them in a single burst! To sing, to scream, to dance barefoot in the woods in the dead of night, with no more awareness of mortality than an animal! These are powerful mysteries. The bellowing of bulls. Springs of honey bubbling from the ground. If we are strong enough in our souls we can rip away the veil and look that naked, terrible beauty right in the face; let God consume us, devour us, unstring our bones. Then spit us out reborn.” \\
\end{enumerate}
\end{minipage}

\begin{minipage}{\textwidth}
\begin{enumerate}[resume=main]
  \item “Being the only female in what was basically a boys’ club must have been difficult for her. Miraculously, she didn’t compensate by becoming hard or quarrelsome. She was still a girl, a slight lovely girl who lay in bed and ate chocolates, a girl whose hair smelled like hyacinth and whose scarves fluttered jauntily in the breeze. But strange and marvelous as she was, a wisp of silk in a forest of black wool, she was not the fragile creature one would have her seem.” \\
\end{enumerate}
\end{minipage}

\begin{minipage}{\textwidth}
\begin{enumerate}[resume=main]
  \item “He cleared his throat. `My movements are restricted,' he said. `I no longer have the ability to travel as freely as I would like.' Hagia Sophia. St. Mark's, in Venice. 'What is this place?' I asked him. `That information is classified, I'm afraid.' I looked around curiously. It seemed that I was the only visitor. `Is it open to the public?' I said. `Not generally, no.' I looked at him. There was so much I wanted to ask him, so much I wanted to say; but somehow I knew there wasn't time and even if there was, that it was all, somehow, beside the point. `Are you happy here?' I said at last. He considered this for a moment. 'Not particularly,' he said. `But you're not very happy where you are, either.' St. Basil's, in Moscow. Chartres. Salisbury and Amiens. He glanced at his watch. `I hope you'll excuse me,' he said, 'but I'm late for an appointment.' He turned from me and walked away. I watched his back receding down the long, gleaming hall.” 
\end{enumerate}
\end{minipage}

\newpage

\textit{Once you've made your answer, put the order in which they are put into and combine them all into one number. (For example; 1, 2, 3, 4 and 5 would become 12345.) Take this number, multiply it by your number from the crossword, and the first three digits of that are the combination code for the lock.}

\section{Choices}

\newpage

\section*{Answer}

\end{document}
